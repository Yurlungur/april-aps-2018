\pdfobjcompresslevel=0
\documentclass[9pt,usepdftitle=false,aspectratio=169]{beamer}
\mode<presentation>

% Settings
% ----------------------------------------------------------------------
% beamer stuff
% Gives us the bottom line with all the goodies
\useoutertheme{infolines}
% Just the theme to use. Should be built into bemaer. Setting the
% height gets rid of a whole lot of whitespace
%\usetheme[height=7mm]{Rochester}
\usetheme[myclass=LA-UR-18-22923,height=7mm,]{LANL-beta}
\usefonttheme{serif}
% Usually beamer gives you navigation hyperlinks on the bottom
% right. I turned this off. It's annoying.
\setbeamertemplate{navigation symbols}{}
% Makes my text boxes look pretty
\setbeamertemplate{blocks}[rounded][shadow=true]
% Makes my bullet points 3d balls
\setbeamertemplate{items}[ball]

% Packages jonah likes
\usepackage{multimedia}
\usepackage{tabularx}
\usepackage{booktabs}
\usepackage{subfigure}
\usepackage{graphicx}
\usepackage{amsmath}
\usepackage{latexsym}
\usepackage{mathrsfs}
\usepackage{amsmath,amssymb,latexsym}
\usepackage[mathscr]{eucal}
\usepackage{mathrsfs}
\usepackage{verbatim}
\usepackage{braket}
\usepackage{listings}
\usepackage{amsthm}
\usepackage{xcolor}
% \usepackage[usenames,dvipsnames,svgnames,table]{xcolor}
\usepackage{fancybox}
\usepackage{animate}
% \usepackage{media9}
\usepackage{multicol}
\usepackage{mdframed}
% \usepackage{scalerel}
\usepackage{hyperref}

% Macros

% Blackboard Bold
\newcommand{\R}{\mathbb{R}}
\newcommand{\Z}{\mathbb{Z}}
\newcommand{\N}{\mathbb{N}}
\newcommand{\Q}{\mathbb{Q}}
\newcommand{\A}{\mathbb{A}}
\newcommand{\E}{\mathbb{E}}
% other
\newcommand{\eval}{\biggr\rvert} %evaluated at
\newcommand{\myvec}[1]{\mathbf{#1}} % vectors for me
% total derivatives
\newcommand{\diff}[2]{\frac{d #1}{d #2}}
\newcommand{\dd}[1]{\frac{d}{d #1}}
% partial derivatives
\newcommand{\pd}[2]{\frac{\partial #1}{\partial #2}}
\newcommand{\pdd}[1]{\frac{\partial}{\partial #1}}
% Order operator
\DeclareRobustCommand{\orderof}{\ensuremath{\mathcal{O}}}
% norm
\newcommand{\norm}[1]{\left| #1 \right|_2}

% tikz
\usepackage{tikz}
\usetikzlibrary{arrows}
\usepackage{pgfplots}

% fix for bug in color.sty
% see: http://tex.stackexchange.com/questions/274524/definecolorset-of-xcolor-problem-with-color-values-starting-with-f
\makeatletter
\def\@hex@@Hex#1%
 {\if a#1A\else \if b#1B\else \if c#1C\else \if d#1D\else
  \if e#1E\else \if f#1F\else #1\fi\fi\fi\fi\fi\fi \@hex@Hex}
\makeatother

%\definecolor{darkgreen}{HTML}{006622}
\definecolor{mypurple}{HTML}{cc00ff}
\definecolor{darkgreen}{HTML}{4f8820}

% Keys to support piece-wise uncovering of elements in TikZ pictures:
% \node[visible on=<2->](foo){Foo}
% \node[visible on=<{2,4}>](bar){Bar}   % put braces around comma expressions
% 
% Internally works by setting opacity=0 when invisible, which has the
% adavantage (compared to \node<2->(foo){Foo} that the node is always there, hence
% always consumes space plus that coordinate (foo) is always available).
% 
% The actual command that implements the invisibility can be overriden
% by altering the style invisible. For instance \tikzsset{invisible/.style={opacity=0.2}}
% would dim the "invisible" parts. Alternatively, the color might be set to white, if the
% output driver does not support transparencies (e.g., PS)
% 
\tikzset{
  invisible/.style={opacity=0},
  visible on/.style={alt={#1{}{invisible}}},
  alt/.code args={<#1>#2#3}{%
    \alt<#1>{\pgfkeysalso{#2}}{\pgfkeysalso{#3}} % \pgfkeysalso doesn't change the path
  },
}

% some nice flowchart features
\tikzset{
  mynode/.style={rectangle,rounded corners,draw=black, top color=white, bottom color=yellow!50,very thick, inner sep=1em, minimum size=3em, text centered},
  myarrow/.style={->, >=latex', shorten >=1pt, thick},
  mylabel/.style={text width=7em, text centered}
}

% define a really nice visible "purple"
\definecolor{gimppurple}{HTML}{AD26FB}
% a light grey
\definecolor{lightgrey}{HTML}{E0E0E0}
% for highlighting
\definecolor{deepblue}{rgb}{0,0,0.5}
\definecolor{deepred}{rgb}{0.6,0,0}
\definecolor{deepgreen}{rgb}{0,0.5,0}
\definecolor{headerblue}{rgb}{0.2,0.2,0.70}

% fonts
% Default fixed font does not support bold face
\DeclareFixedFont{\ttb}{T1}{txtt}{bx}{n}{10} % for bold
\DeclareFixedFont{\ttm}{T1}{txtt}{m}{n}{10}  % for normal

% Python style for highlighting
\newcommand\pythonstyle{\lstset{
    language=Python,
    basicstyle=\footnotesize\ttm,
    otherkeywords={self},
    keywordstyle=\ttb\color{deepblue},
    emph={__init__},
    emphstyle=\ttb\color{deepred},
    commentstyle=\ttfamily\color{deepred},
    stringstyle=\color{deepgreen},
    frame=tb,
    showstringspaces=false
  }}

% defines the algorithm listing environment
\newcommand{\pseudocodestyle}{
  \lstset{
    basicstyle=\footnotesize\ttm,
    keywordstyle=\ttb\color{deepblue},
    emphstyle=\ttb\color{deepred},
    commentstyle=\ttfamily\color{deepred},
    stringstyle=\color{deepgreen},
    frame=tb,
    showstringspaces=false
    % emph={} % these are bold
    % comments character
    morecomment=[l]{\# } % pick a comment character
    keywords={,input, output, return, datatype, function, in, if, else, foreach, while, begin, end, }
    % numbers=left, % if you want numbering
    % xleftmargin=.04\textwidth,
  }
}
\lstnewenvironment{pseudocode}[1][]
{
  \peudocodestyle
  \lstset{#1}
}
{}

% Python environment
\lstnewenvironment{python}[1][]
{
  \pythonstyle
  \lstset{#1}
}
{}

% Hide headline for a Beamer slide
\makeatletter
    \newenvironment{withoutheadline}{
        \setbeamertemplate{headline}[default]
        \def\beamer@entrycode{\vspace*{-\headheight}}
    }{}
\makeatother

% Backup slides. Makes the slide numbering
% stop at last "presentation slide"
% but allows for backup slides after that.
\newcommand{\backupbegin}{ % call this before the first backup slide
  \newcounter{finalframe}
  \setcounter{finalframe}{\value{framenumber}}
}
\newcommand{\backupend}{ % call this after the last backupslide
  \setcounter{framenumber}{\value{finalframe}}
}

% a macro for a useful transition slide
\newcommand{\transitionslide}[1]{
  \begin{frame}[plain]
    \vfill
    \centering
    \begin{beamercolorbox}[sep=8pt,center,shadow=true,rounded=true]{title}
      \usebeamerfont{title}{#1}
    \end{beamercolorbox}
    \vfill
  \end{frame}
}

% Graphics go in the Figures folder
\graphicspath{{Figures/}}
% ----------------------------------------------------------------------


% Preamble
% ----------------------------------------------------------------------
\title[Spectral Discontinuities]{Spectral Methods in the Presence of Discontinuities}

\author[Co-Design Summer School]{
  % See: http://tex.stackexchange.com/questions/45938/error-when-using-colour-in-author
  J. M. Miller,
  {\texorpdfstring{\color{blue}}{}J. Piotrowska},
  E. Schnetter}

\institute[LANL]{\color{blue}Los Alamos National Laboratory}

\date[April APS]{\color{black}April APS, 2018}
\LAUR{LA-UR-18-22923}
% ----------------------------------------------------------------------


% The actual document
% ----------------------------------------------------------------------
\begin{document}

\section{Introduction}
\begin{frame}[plain]
  \titlepage
\end{frame}

\begin{frame}
  \frametitle{Spectral Methods}
  \begin{columns}
    \begin{column}{0.5\textwidth}
      \begin{center}
        \underline{\huge\color{blue}The Basic Idea}\\
        \vspace{0.75cm}
        \begin{huge}
          $$u({\color{red}t},{\color{blue}x}) = \sum_{i=0}^N{\color{red}\hat{u}_i(t)}{\color{blue}\phi_i(x)}$$
        \end{huge}
        \vspace{2cm}
      \end{center}
    \end{column}
    \vrule{}
    \begin{column}{0.05\textwidth}\end{column}
    \begin{column}{0.45\textwidth}
      \begin{center}
        \underline{\huge\color{red}Details}
      \end{center}
      \vspace{1.2cm}
      \begin{itemize}
      \item We use \textit{Method of lines}
      \item We use \textit{pseudospectral} methods
      \item We use the \textit{Chebyshev basis}
      \item We solve the \textit{advection equation}
        { \huge
          $$\partial_t u + \partial_x (c u) = 0 $$
        }
      \end{itemize}
      \vspace{1cm}
    \end{column}
  \end{columns}
\end{frame}

\begin{frame}
  \frametitle{Convergence for Smooth Functions}
  \begin{Large}
  \begin{displaymath}
    \norm{u - S_N[u]} \leq \frac{\alpha}{N^{N}}
  \end{displaymath}
  \end{Large}
  \begin{columns}
    \begin{column}{0.5\textwidth}
      \begin{center}
        \includegraphics[width=0.9\textwidth]{nogibbs}
      \end{center}
    \end{column}
    \begin{column}{0.5\textwidth}
      \begin{center}
        \includegraphics[width=\textwidth]{gaussian_error}
      \end{center}
    \end{column}
  \end{columns}
\end{frame}

\begin{frame}
  \frametitle{Spectral Methods in Relativity}
  \setlength{\unitlength}{1cm}
  \begin{picture}(12,8)
    \put(4.5,0.75){\includegraphics[width=7cm]{spec-domain-decomp}}
    \put(11.5,1.00){Kidder et al., 2000}
    \put(-0.75,1.2){\includegraphics[height=4cm]{gauge_wave_3D_slice_lowres}}
    \put(0.,1.0){Miller and Schnetter, 2016}
    \put(10,4.5){\includegraphics[height=3.5cm]{ansorg_brugmann_tichy_twopunctures}}
    \put(10.5,8.){Ansorg et al., 2004}
    \put(3.,3.5){\includegraphics[clip,trim={10cm 0 10cm 0},width=4cm]{BBH_gravitational_lensing.jpg}}
    \put(0.5,7.5){Bohn et al., 2015}
  \end{picture}
\end{frame}

\begin{frame}
  \frametitle{Convergence for Non-Smooth Functions}
  \begin{Large}
  \begin{displaymath}
    \norm{u - S_N[u]} \leq \frac{\alpha}{N^{\color{red}m}}\sum_{k=0}^{{\color{red}m}}
    \norm{f^{(k)}},\ f\in \mathcal{C}^{\color{red}m}
  \end{displaymath}
  \end{Large}
  \begin{columns}
    \begin{column}{0.5\textwidth}
      \begin{center}
        \includegraphics[width=0.9\textwidth]{gibbs}
      \end{center}
    \end{column}
    \begin{column}{0.5\textwidth}
      \begin{center}
        \includegraphics[width=0.9\textwidth]{tophat_error}
      \end{center}
    \end{column}
  \end{columns}
\end{frame}

\begin{frame}
  \frametitle{Solution: One-Sided {\color{blue}Mollifiers}}
  \begin{large}
  \begin{eqnarray}
    {\color{blue}\Phi_{p,\delta}}*({\color{red}S_N[f]})(x) = \int {\color{blue}G(x,y)} {\color{red}S_N[f](y)}dy\nonumber\\
    \lvert {\color{blue}\Phi^d_{p_N}}*({\color{red}S_N[f]})(x) - f(x) \rvert \lesssim e ^{- \eta d N}\nonumber
  \end{eqnarray}
  \end{large}
  \begin{center}
    \includegraphics[width=\textwidth]{mollifiers}\\
    {\small Based on work in Tadmor, 2002}
  \end{center}
\end{frame}

\begin{frame}
  \frametitle{Mollified Step Function}
  \begin{columns}
    \begin{column}{0.4\textwidth}
      \centering
      \includegraphics[width=\textwidth]{mollifiedN32_edge}
    \end{column}
    \begin{column}{0.7\textwidth}
      \centering
      \includegraphics[width=\textwidth]{mollified_unmollified}
    \end{column}
  \end{columns}
\end{frame}

\begin{frame}
  \frametitle{Mollified Boxcar}
  \begin{columns}
    \begin{column}{0.5\textwidth}
      \includegraphics[width=\textwidth]{mollifiedN32_centre}
    \end{column}
    \begin{column}{0.5\textwidth}
        \includegraphics[width=\textwidth]{mollified_pointwise_error}
        %\includegraphics[width=\textwidth]{mollifier_convergence_boundary}
    \end{column}
  \end{columns}
\end{frame}

\begin{frame}
  \frametitle{Pointwise Convergence}
  \begin{columns}
    \begin{column}{0.35\textwidth}
      \includegraphics[width=\textwidth]{mollifier_convergence_boundary}
      %\includegraphics[width=\textwidth]{mollifiedN32_centre}
    \end{column}
    \begin{column}{0.75\textwidth}
      \includegraphics[width=\textwidth]{mollifier_convergence_x0}
        %\includegraphics[width=\textwidth]{mollified_pointwise_error}
    \end{column}
  \end{columns}
\end{frame}

\begin{frame}
  \frametitle{Convergence Preserved in Time}
  \begin{columns}
    \begin{column}{0.5\textwidth}
      \begin{center}
        \underline{\huge\color{blue}Procedure}
      \end{center}
      \begin{itemize}
      \item Use pseudospectral discretization and method of lines to
        solve PDE
      \item Mollify solution \textit{as postprocessing step only}
      \item High-order convergence is achieved in time.
      \end{itemize}
    \end{column}
    \begin{column}{0.5\textwidth}
      \includegraphics[height=0.95\textheight]{movie/frame_0001}
      % \animategraphics[height=0.95\textheight,every=7,autoplay,loop]
      % {7}{movie/frame_}{0001}{2000}
    \end{column}
  \end{columns}
\end{frame}

\begin{frame}
  \frametitle{Conclusions and Future Work}
  \begin{columns}
    \begin{column}{.5\textwidth}
      \begin{itemize}
      \item Spectral methods are very powerful
        \begin{itemize}
        \item Exponential convergence for smooth problems
        \item For non-smooth problems, convergence spoiled by Gibbs
        \end{itemize}
      \item If discontinuity locations are known, exponential
        convergence can be recovered far from the discontinuity via
        \textit{mollifiers}
      \item We achieve poor convergence near the discontinuity, but
        exponential convergence far from it.
      \end{itemize}
    \end{column}
    \begin{column}{.5\textwidth}
      \begin{itemize}
      \item Open Questions:
        \begin{itemize}
        \item How does nonlinearity influence these results?
        \item How do artificial atmospheres?
        \item How to make the procedure efficient?
        \item How robust is the procedure in multiple dimensions?
        \end{itemize}
      \end{itemize}
    \end{column}
  \end{columns}
  \vspace{1cm}
  \begin{itemize}
    \item For more details, see
      {\color{blue}\href{https://arxiv.org/abs/1712.09952}{arXiv:1712.09952}}
    \end{itemize}
\end{frame}

\backupbegin

\begin{frame}
  \frametitle{Edge Detection}
  \begin{columns}
    \begin{column}{0.5\textwidth}
      \begin{Large}
        \begin{displaymath}
          \frac{\pi\sqrt{1-x^2}}{N}\sum^N_{k=1} {\color{blue}\mu\Bigg(\frac{k}{N}\Bigg)}
          \hat{f}(k){T^\prime}_k(x) \longrightarrow {\color{red}[f](x)}
        \end{displaymath}
      \end{Large}
      \vspace{0.025\textwidth}
      \begin{center}
        \includegraphics[width=0.9\textwidth]{minmod_good}
      \end{center}
      {\small Based on work in Gelb, 1999}
    \end{column}
    \begin{column}{0.5\textwidth}
      \begin{center}
        \includegraphics[width=0.85\textwidth]{jump_trig}\\
        \includegraphics[width=0.85\textwidth]{jump_poly}\\
        \includegraphics[width=0.85\textwidth]{jump_expon}\\
      \end{center}
    \end{column}
  \end{columns}
\end{frame}

\backupend

\end{document}
